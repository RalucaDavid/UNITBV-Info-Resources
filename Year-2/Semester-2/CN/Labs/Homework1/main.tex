% LaTeX Template For MATH 490 @ VCU
\documentclass{article}
\usepackage{hyperref}
\usepackage{amsmath}
\usepackage{amsthm}
\usepackage{amssymb}

%%%%%%%%%% EXACT 1in MARGINS + DOUBEL SPACED %%%%%%%
% NOTE IF YOU USE 1IN MARGINS CHANGE THE FONT 	  %%
% SIZE TO 12PT IN THE FIRST LINE OF THIS DOCUMENT %%
%\linespread{2}									  %%
%\setlength{\textwidth}{6.5in}     				  %%
%\setlength{\oddsidemargin}{0in}   				  %% 
%\setlength{\evensidemargin}{0in}  				  %%
%\setlength{\textheight}{8.5in}    				  %%
%\setlength{\topmargin}{0in}      				  %%
%\setlength{\headheight}{0in}     				  %%
%\setlength{\headsep}{0in}       				  %%
%\setlength{\footskip}{.5in}    				  %%
%%%%%%%%%%%%%%%%%%%%%%%%%%%%%%%%%%%%%%%%%%%%%%%%%%%%



\begin{document}

\title{Calcul numeric - tem\u{a} de laborator}

\author{}

\date{Februarie - Mai 2024}

\maketitle              % typeset the title of the contribution


% You don't need an abstract or keywords for an article review
%\begin{abstract}
%The abstract should summarize the contents of the paper
%using at least 70 and at most 150 words. It will be set in 9-point
%font size and be inset 1.0 cm from the right and left margins.
%There will be two blank lines before and after the Abstract. \dots
%\keywords{List up to three keywords here, like this:
%computational geometry, graph theory, Hamilton cycles}
%\end{abstract}


% TO MAKE A TITLE PAGE USE THE FOLLOWING COMMAND HERE.
% \newpage




\section*{Enun\c{t}: Capitolul 8, Subcapitolul I, Problema 1}

S\u{a} se calculeze polinoamele de aproximare de grad unu, doi si trei construite prin metoda celor mai mici pătrate pentru datele Problemei 1 din capitolul Interpolare.
\begin{center}
$
A=\begin{pmatrix}
10 & 6 & -2 & 1\\
10 & 10 & -5 & 0\\
-2 & 2 & -2 & 1\\
1 & 3 & -2 & 3
\end{pmatrix}
$
\end{center}

\section*{Solu\c{t}ie}

\begin{center}
\begin{enumerate}
    \item 
    Declarăm matricea A. \\
    \begin{center}
    A = [10,6,-2,1; 10,10,-5,0; -2,2,-2,1; 1,3,-2,3];
    \end{center}
    \item 
    Apelăm funcția lu pentru calcularea factorizării LU a matricei A. \\
    \begin{center}
    [L,U,P]=lu(A)
    \end{center}
\end{enumerate}
\end{center}

\section*{Rezultat}
\begin{center}
$
L =\begin{pmatrix}

    1.0000  &       0  &       0  &       0\\
    1.0000  &  1.0000  &       0  &       0\\
    -0.2000  &  0.8000  &  1.0000  &       0\\
    0.1000  &  0.6000  &  -0.5000  &  1.0000
\end{pmatrix}
$
\end{center}

\begin{center}
$
U =\begin{pmatrix}
  10.0000  &  6.0000  &  -2.0000  &  1.0000\\
    0  &  4.0000 & -3.0000 &  -1.0000\\
    0  & 0  &  0.0000  & 2.0000\\
    0  & 0  &  0  &  4.5000
\end{pmatrix}
$
\end{center} 

\begin{center}
$
P =\begin{pmatrix}
  1  &   0  &   0   & 0\\
  0  &   1  &   0   &  0\\
  0  &   0  &   1   &  0\\
  0  &   0  &   0   &  1
\end{pmatrix}
$
\end{center}

\end{document}
