% LaTeX Template For MATH 490 @ VCU
\documentclass{article}
\usepackage{hyperref}
\usepackage{amsmath}
\usepackage{amsthm}
\usepackage{amssymb}
\usepackage{xcolor}

%%%%%%%%%% EXACT 1in MARGINS + DOUBEL SPACED %%%%%%%
% NOTE IF YOU USE 1IN MARGINS CHANGE THE FONT 	  %%
% SIZE TO 12PT IN THE FIRST LINE OF THIS DOCUMENT %%
%\linespread{2}									  %%
%\setlength{\textwidth}{6.5in}     				  %%
%\setlength{\oddsidemargin}{0in}   				  %% 
%\setlength{\evensidemargin}{0in}  				  %%
%\setlength{\textheight}{8.5in}    				  %%
%\setlength{\topmargin}{0in}      				  %%
%\setlength{\headheight}{0in}     				  %%
%\setlength{\headsep}{0in}       				  %%
%\setlength{\footskip}{.5in}    				  %%
%%%%%%%%%%%%%%%%%%%%%%%%%%%%%%%%%%%%%%%%%%%%%%%%%%%%



\begin{document}

\title{Calcul numeric - tem\u{a} de laborator}

\author{}

\date{Februarie - Mai 2024}

\maketitle              % typeset the title of the contribution


% You don't need an abstract or keywords for an article review
%\begin{abstract}
%The abstract should summarize the contents of the paper
%using at least 70 and at most 150 words. It will be set in 9-point
%font size and be inset 1.0 cm from the right and left margins.
%There will be two blank lines before and after the Abstract. \dots
%\keywords{List up to three keywords here, like this:
%computational geometry, graph theory, Hamilton cycles}
%\end{abstract}


% TO MAKE A TITLE PAGE USE THE FOLLOWING COMMAND HERE.
% \newpage




\section*{Enun\c{t}: Capitolul 11, Subcapitolul I, Problema 3}

S\u{a} se calculeze derivate funcțiilor în punctul indicat:

\begin{center}
f(x) = x - 2\cos(x), \quad $x_0$=1

\end{center}

\section*{Solu\c{t}ie}

\begin{center}
\begin{enumerate}
\item  Definim funcția f. \\
\begin{center}
    syms x;\\
    f = x - 2*cos(x);
\end{center}
\item Calculăm derivata de ordinul I al funcției f. \\
 \begin{center}
    f\_dev\_first = diff(f, x);
 \end{center}
\item Calculăm derivata de ordinul II al funcției f. \\
 \begin{center}
    f\_dev\_second = diff(f\_dev\_first, x);
 \end{center}
 \item Calculăm derivata de ordinul I în punctul $x_0$.\\
 \begin{center}
    f\_dev\_first\_value = double(subs(f\_dev\_first, x, x0));
 \end{center}
  \item Calculăm derivata de ordinul II în punctul $x_0$. \\
 \begin{center}
    f\_dev\_second\_value = double(subs(f\_dev\_second, x, x0));
 \end{center}
\end{enumerate}
\end{center}

\section*{Rezultat}
\begin{center}
f\_dev\_first\_value = 2.6829 \\ f\_dev\_second\_value = 1.0806
\end{center}
\end{document}
