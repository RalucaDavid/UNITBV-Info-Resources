% LaTeX Template For MATH 490 @ VCU
\documentclass{article}
\usepackage{hyperref}
\usepackage{amsmath}
\usepackage{amsthm}
\usepackage{amssymb}

%%%%%%%%%% EXACT 1in MARGINS + DOUBEL SPACED %%%%%%%
% NOTE IF YOU USE 1IN MARGINS CHANGE THE FONT 	  %%
% SIZE TO 12PT IN THE FIRST LINE OF THIS DOCUMENT %%
%\linespread{2}									  %%
%\setlength{\textwidth}{6.5in}     				  %%
%\setlength{\oddsidemargin}{0in}   				  %% 
%\setlength{\evensidemargin}{0in}  				  %%
%\setlength{\textheight}{8.5in}    				  %%
%\setlength{\topmargin}{0in}      				  %%
%\setlength{\headheight}{0in}     				  %%
%\setlength{\headsep}{0in}       				  %%
%\setlength{\footskip}{.5in}    				  %%
%%%%%%%%%%%%%%%%%%%%%%%%%%%%%%%%%%%%%%%%%%%%%%%%%%%%



\begin{document}

\title{Calcul numeric - tem\u{a} de laborator}

\author{}

\date{Februarie - Mai 2024}

\maketitle              % typeset the title of the contribution


% You don't need an abstract or keywords for an article review
%\begin{abstract}
%The abstract should summarize the contents of the paper
%using at least 70 and at most 150 words. It will be set in 9-point
%font size and be inset 1.0 cm from the right and left margins.
%There will be two blank lines before and after the Abstract. \dots
%\keywords{List up to three keywords here, like this:
%computational geometry, graph theory, Hamilton cycles}
%\end{abstract}


% TO MAKE A TITLE PAGE USE THE FOLLOWING COMMAND HERE.
% \newpage




\section*{Enun\c{t}: Capitolul 8, Subcapitolul II, Problema 3}

S\u{a} se calculeze descompunerea / factorizarea QR a matricei:
\begin{center}
$
A=\begin{pmatrix}
3 & 4 & 7 & -2\\
5 & 4 & 9 & 3\\
1 & -1 & 0 & 3\\
1 & -1 & 0 & 0
\end{pmatrix}
$
\end{center}

\section*{Solu\c{t}ie}

\begin{center}
\begin{enumerate}
    \item 
    Declarăm matricea A. \\
    \begin{center}
    A=[3,4,7,-2;5,4,9,3;1,-1,0,3;1,-1,0,0];
    \end{center}
    \item 
    Apelăm funcția qr pentru calcularea descompunerii / factorizării QR a matricei A. \\
    \begin{center}
    [Q,R,P]=qr(A)
    \end{center}
\end{enumerate}
\end{center}

\section*{Rezultat}
\begin{center}
$
Q =\begin{pmatrix}

    -0.6139  &  0.5934 & -0.1446  &       -0.5000\\
    -0.7894  &  -0.4616  & 0.1125 &        0.3889\\
          0  &  -0.6594  &  -0.2088  &  -0.7222\\
          0  &       0  &   -0.9606  &   0.2778
\end{pmatrix}
$
\end{center}

\begin{center}
$
R =\begin{pmatrix}
  -11.4018  &    -1.1402  &  -5.6132  &  -5.7886\\
    0  &  -4.5497& 1.1869 &  -1.1869\\
    0  & 0  &  1.0410  & -1.0410\\
    0  & 0  &  0  &  0.0000
\end{pmatrix}
$
\end{center} 

\begin{center}
$
P =\begin{pmatrix}
  0  &   0  &   0   & 1\\
  0  &   0  &   1   &  0\\
  1  &   0  &   0   &  0\\
  0  &   1  &   0   &  0
\end{pmatrix}
$
\end{center}

\end{document}
