% LaTeX Template For MATH 490 @ VCU
\documentclass{article}
\usepackage{hyperref}
\usepackage{amsmath}
\usepackage{amsthm}
\usepackage{amssymb}
\usepackage{xcolor}

%%%%%%%%%% EXACT 1in MARGINS + DOUBEL SPACED %%%%%%%
% NOTE IF YOU USE 1IN MARGINS CHANGE THE FONT 	  %%
% SIZE TO 12PT IN THE FIRST LINE OF THIS DOCUMENT %%
%\linespread{2}									  %%
%\setlength{\textwidth}{6.5in}     				  %%
%\setlength{\oddsidemargin}{0in}   				  %% 
%\setlength{\evensidemargin}{0in}  				  %%
%\setlength{\textheight}{8.5in}    				  %%
%\setlength{\topmargin}{0in}      				  %%
%\setlength{\headheight}{0in}     				  %%
%\setlength{\headsep}{0in}       				  %%
%\setlength{\footskip}{.5in}    				  %%
%%%%%%%%%%%%%%%%%%%%%%%%%%%%%%%%%%%%%%%%%%%%%%%%%%%%



\begin{document}

\title{Calcul numeric - tem\u{a} de laborator}

\author{}

\date{Februarie - Mai 2024}

\maketitle              % typeset the title of the contribution


% You don't need an abstract or keywords for an article review
%\begin{abstract}
%The abstract should summarize the contents of the paper
%using at least 70 and at most 150 words. It will be set in 9-point
%font size and be inset 1.0 cm from the right and left margins.
%There will be two blank lines before and after the Abstract. \dots
%\keywords{List up to three keywords here, like this:
%computational geometry, graph theory, Hamilton cycles}
%\end{abstract}


% TO MAKE A TITLE PAGE USE THE FOLLOWING COMMAND HERE.
% \newpage




\section*{Enun\c{t}: Capitolul 9, Subcapitolul I, Problema 7}

S\u{a} se rezolve ecuațiile polinomiale:

\begin{center}
\[
x^5 - 12x^4 + 50x^3 - 88x^2 + 96x - 128 = 0
\]
\end{center}

\section*{Solu\c{t}ie}

\begin{center}
\begin{enumerate}
\item Declarăm vectorul coeficienților ecuației polinomiale. \\
 \begin{center}
    coeficienti = [1, -12, 50, -88, 96, -128];
    \end{center}
\item Calculăm rădăcinile ecuației polinomiale prin apelarea funcției roots. \\
\begin{center}
    radacinile = roots(coeficienti);
    \end{center}
\end{enumerate}
\end{center}

\section*{Rezultat}
\begin{center}
$x\textsubscript{1}$ = 4.0000 + 0.0000i \\
$x\textsubscript{2}$ = 4.0000 - 0.0000i \\
$x\textsubscript{3}$ = 4.0000 + 0.0000i \\
$x\textsubscript{4}$ = -0.0000 + 1.4142i \\
$x\textsubscript{5}$ = -0.0000 - 1.4142i
\end{center}

\end{document}
