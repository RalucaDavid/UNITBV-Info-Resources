% LaTeX Template For MATH 490 @ VCU
\documentclass{article}
\usepackage{hyperref}
\usepackage{amsmath}
\usepackage{amsthm}
\usepackage{amssymb}
\usepackage{xcolor}

%%%%%%%%%% EXACT 1in MARGINS + DOUBEL SPACED %%%%%%%
% NOTE IF YOU USE 1IN MARGINS CHANGE THE FONT 	  %%
% SIZE TO 12PT IN THE FIRST LINE OF THIS DOCUMENT %%
%\linespread{2}									  %%
%\setlength{\textwidth}{6.5in}     				  %%
%\setlength{\oddsidemargin}{0in}   				  %% 
%\setlength{\evensidemargin}{0in}  				  %%
%\setlength{\textheight}{8.5in}    				  %%
%\setlength{\topmargin}{0in}      				  %%
%\setlength{\headheight}{0in}     				  %%
%\setlength{\headsep}{0in}       				  %%
%\setlength{\footskip}{.5in}    				  %%
%%%%%%%%%%%%%%%%%%%%%%%%%%%%%%%%%%%%%%%%%%%%%%%%%%%%



\begin{document}

\title{Calcul numeric - tem\u{a} de laborator}

\author{}

\date{Februarie - Mai 2024}

\maketitle              % typeset the title of the contribution


% You don't need an abstract or keywords for an article review
%\begin{abstract}
%The abstract should summarize the contents of the paper
%using at least 70 and at most 150 words. It will be set in 9-point
%font size and be inset 1.0 cm from the right and left margins.
%There will be two blank lines before and after the Abstract. \dots
%\keywords{List up to three keywords here, like this:
%computational geometry, graph theory, Hamilton cycles}
%\end{abstract}


% TO MAKE A TITLE PAGE USE THE FOLLOWING COMMAND HERE.
% \newpage




\section*{Enun\c{t}: Capitolul 9, Subcapitolul II, Problema 9}

S\u{a} se rezolve  ecuațiile algebrice:

\begin{center}
\[
x^x +2x -6  = 0
\]
\end{center}

\section*{Solu\c{t}ie}

\begin{center}
\begin{enumerate}
\item Rescriem ecuația în altă formă. \\
 \begin{center}
    \[
x^x +2x -6  = 0  
\] 
\[
x^x = 6-2x / \ln
\] 
\[
x\ln(x) = \ln(6-2x)
\] 
    \end{center}
\item Definim ecuația sub formă de funcție anonimă. \\
\begin{center}
    equation = @(x) x .* log(x) - log(6 - 2*x);
    \end{center}
\item Definim un punct de start. \\
\begin{center}
    x0 = 1;
    \end{center}
\item Rezolvăm ecuația prin apelarea funcției fsolve. \\
\begin{center}
    sol = fsolve(equation, x0);
    \end{center}
\end{enumerate}
\end{center}

\section*{Rezultat}
\begin{center}
   x = 1.7231;
\end{center}
\section*{Observație}
A trebuit să instalez \textbf{Optimization Toolbox} pentru a folosi funcția fsolve.
\end{document}
