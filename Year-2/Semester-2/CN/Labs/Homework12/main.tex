% LaTeX Template For MATH 490 @ VCU
\documentclass{article}
\usepackage{hyperref}
\usepackage{amsmath}
\usepackage{amsthm}
\usepackage{amssymb}
\usepackage{xcolor}

%%%%%%%%%% EXACT 1in MARGINS + DOUBEL SPACED %%%%%%%
% NOTE IF YOU USE 1IN MARGINS CHANGE THE FONT 	  %%
% SIZE TO 12PT IN THE FIRST LINE OF THIS DOCUMENT %%
%\linespread{2}									  %%
%\setlength{\textwidth}{6.5in}     				  %%
%\setlength{\oddsidemargin}{0in}   				  %% 
%\setlength{\evensidemargin}{0in}  				  %%
%\setlength{\textheight}{8.5in}    				  %%
%\setlength{\topmargin}{0in}      				  %%
%\setlength{\headheight}{0in}     				  %%
%\setlength{\headsep}{0in}       				  %%
%\setlength{\footskip}{.5in}    				  %%
%%%%%%%%%%%%%%%%%%%%%%%%%%%%%%%%%%%%%%%%%%%%%%%%%%%%



\begin{document}

\title{Calcul numeric - tem\u{a} de laborator}

\author{}

\date{Februarie - Mai 2024}

\maketitle              % typeset the title of the contribution


% You don't need an abstract or keywords for an article review
%\begin{abstract}
%The abstract should summarize the contents of the paper
%using at least 70 and at most 150 words. It will be set in 9-point
%font size and be inset 1.0 cm from the right and left margins.
%There will be two blank lines before and after the Abstract. \dots
%\keywords{List up to three keywords here, like this:
%computational geometry, graph theory, Hamilton cycles}
%\end{abstract}


% TO MAKE A TITLE PAGE USE THE FOLLOWING COMMAND HERE.
% \newpage




\section*{Enun\c{t}: Capitolul 11, Subcapitolul II, Problema 5}

S\u{a} se calculeze jacobianul și hessianul funcțiilor în punctul indicat:
\[
f(x, y) = \begin{cases}
x^3 + y^3 - 6x + 3  & (x,y) = (1,2) \\
x^3 - y^3 - 6y + 2 
\end{cases}
\]

\section*{Solu\c{t}ie}

\begin{center}
\begin{enumerate}
\item  Definim funcția f. \\
\begin{center}
     syms x y; \\
     f1 = $x^3$ + $y^3$ - 6*x + 3; \\
     f2 = $x^3$ - $y^3$ - 6*y + 2;
\end{center}
\item Calculăm jacobianul prin apelarea funcției jacobian. \\
 \begin{center}
    jacobian\_matrix = jacobian([f1, f2], [x, y]);
 \end{center}
\item Calculăm jacobianul în punctul (1,2). \\
 \begin{center}
    jacobian\_at\_point = double(subs(jacobian\_matrix, [x, y], [1, 2]));
 \end{center}
 \item Calculăm hessianul fiecărei funcții.\\
 \begin{center}
    hessian\_matrix\_f1 = hessian(f1, [x, y]); \\
    hessian\_matrix\_f2 = hessian(f2, [x, y]);
 \end{center}
  \item Calculăm hessianul în punctul (1,2). \\
 \begin{center}
    hessian\_at\_point\_f1 = double(subs(hessian\_matrix\_f1, [x, y], [1, 2]));
    hessian\_at\_point\_f2 = double(subs(hessian\_matrix\_f2, [x, y], [1, 2]));
 \end{center}
\end{enumerate}
\end{center}

\section*{Rezultat}
\begin{center}
  jacobian\_at\_point=
\begin{pmatrix}
-3 & 12\\
 3 & -18 
\end{pmatrix} \\
hessian\_at\_point\_f1=
\begin{pmatrix}
6 & 0\\
 0 & 12
\end{pmatrix} \\
hessian\_at\_point\_f2=
\begin{pmatrix}
6 & 0\\
 0 & -12
\end{pmatrix} 
\end{center}
\end{document}
