% LaTeX Template For MATH 490 @ VCU
\documentclass{article}
\usepackage{hyperref}
\usepackage{amsmath}
\usepackage{amsthm}
\usepackage{amssymb}
\usepackage{xcolor}

%%%%%%%%%% EXACT 1in MARGINS + DOUBEL SPACED %%%%%%%
% NOTE IF YOU USE 1IN MARGINS CHANGE THE FONT 	  %%
% SIZE TO 12PT IN THE FIRST LINE OF THIS DOCUMENT %%
%\linespread{2}									  %%
%\setlength{\textwidth}{6.5in}     				  %%
%\setlength{\oddsidemargin}{0in}   				  %% 
%\setlength{\evensidemargin}{0in}  				  %%
%\setlength{\textheight}{8.5in}    				  %%
%\setlength{\topmargin}{0in}      				  %%
%\setlength{\headheight}{0in}     				  %%
%\setlength{\headsep}{0in}       				  %%
%\setlength{\footskip}{.5in}    				  %%
%%%%%%%%%%%%%%%%%%%%%%%%%%%%%%%%%%%%%%%%%%%%%%%%%%%%



\begin{document}

\title{Calcul numeric - tem\u{a} de laborator}

\author{}

\date{Februarie - Mai 2024}

\maketitle              % typeset the title of the contribution


% You don't need an abstract or keywords for an article review
%\begin{abstract}
%The abstract should summarize the contents of the paper
%using at least 70 and at most 150 words. It will be set in 9-point
%font size and be inset 1.0 cm from the right and left margins.
%There will be two blank lines before and after the Abstract. \dots
%\keywords{List up to three keywords here, like this:
%computational geometry, graph theory, Hamilton cycles}
%\end{abstract}


% TO MAKE A TITLE PAGE USE THE FOLLOWING COMMAND HERE.
% \newpage




\section*{Enun\c{t}: Capitolul 9, Subcapitolul III, Problema 3}

S\u{a} se rezolve sistemele algebrice de ecua¸tii neliniare:

\begin{center}
\begin{align*}
\begin{cases}
e^{-x^y}=x^2-y+1\\
(x+0.5)^2+y^2=0.6\\
 \end{cases}
\end{align*}
\end{center}

\section*{Solu\c{t}ie}

\begin{center}
\begin{enumerate}
\item Definim funcția anonimă care calculează rezultatul sistemului de ecuații ( x=$x\textsubscript{1}$, y=$x\textsubscript{2}$ ). \\
 \begin{center}
    \[
    equations = @(x) [exp(-x(1)\^{}x(2)) - x(1)\^{}2 + x(2) - 1;
                  (x(1)+0.5)\^{}2+ x(2)\^{}2 - 0.6];
\] 
    \end{center}
\item  Alegem o soluție inițială (aproximativă). \\
\begin{center}
    x0 = [0; 0];
    \end{center}
\item Rezolvăm sistemul prin apelarea funcției fsolve. \\
\begin{center}
    x = fsolve(equations, x0);
    \end{center}
\end{enumerate}
\end{center}

\section*{Rezultat}
\begin{center}
   $x\textsubscript{1}$  = 0.16037; $x\textsubscript{2}$  = 0.40486
\end{center}
\end{document}
