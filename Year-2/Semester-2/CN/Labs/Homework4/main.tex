% LaTeX Template For MATH 490 @ VCU
\documentclass{article}
\usepackage{hyperref}
\usepackage{amsmath}
\usepackage{amsthm}
\usepackage{amssymb}
\usepackage{xcolor}

%%%%%%%%%% EXACT 1in MARGINS + DOUBEL SPACED %%%%%%%
% NOTE IF YOU USE 1IN MARGINS CHANGE THE FONT 	  %%
% SIZE TO 12PT IN THE FIRST LINE OF THIS DOCUMENT %%
%\linespread{2}									  %%
%\setlength{\textwidth}{6.5in}     				  %%
%\setlength{\oddsidemargin}{0in}   				  %% 
%\setlength{\evensidemargin}{0in}  				  %%
%\setlength{\textheight}{8.5in}    				  %%
%\setlength{\topmargin}{0in}      				  %%
%\setlength{\headheight}{0in}     				  %%
%\setlength{\headsep}{0in}       				  %%
%\setlength{\footskip}{.5in}    				  %%
%%%%%%%%%%%%%%%%%%%%%%%%%%%%%%%%%%%%%%%%%%%%%%%%%%%%



\begin{document}

\title{Calcul numeric - tem\u{a} de laborator}

\author{}

\date{Februarie - Mai 2024}

\maketitle              % typeset the title of the contribution


% You don't need an abstract or keywords for an article review
%\begin{abstract}
%The abstract should summarize the contents of the paper
%using at least 70 and at most 150 words. It will be set in 9-point
%font size and be inset 1.0 cm from the right and left margins.
%There will be two blank lines before and after the Abstract. \dots
%\keywords{List up to three keywords here, like this:
%computational geometry, graph theory, Hamilton cycles}
%\end{abstract}


% TO MAKE A TITLE PAGE USE THE FOLLOWING COMMAND HERE.
% \newpage




\section*{Enun\c{t}: Capitolul 8, Subcapitolul IV, Problema 1}

S\u{a} se rezolve sistemele algebrice de ecuații liniare incompatibile în sensul
celor mai mici pătrate:
\begin{center}
\begin{align*}
\begin{cases}
3$x\textsubscript{1}$ - $x\textsubscript{2}$ + $x\textsubscript{3}$ &= 3 \\
$x\textsubscript{1}$ + 2$x\textsubscript{2}$ - $x\textsubscript{3}$ &= -4 \\
2$x\textsubscript{1}$ - 3$x\textsubscript{2}$ + 2$x\textsubscript{3}$ &= 3 \\
 \end{cases}
\end{align*}
\end{center}

\section*{Solu\c{t}ie}

\begin{center}
\begin{enumerate}
\item Declarăm matricea coeficienților A. \\
 \begin{center}
    A = [3,-1,1;1,2,-1;2,-3,2];
    \end{center}
\item Declarăm vectorul termenilor liberi b. \\
\begin{center}
    b = [3;-4;3];
    \end{center}
\item Calculăm soluția celui mai mic pătrat prin apelarea funcției lscov.\\
\begin{center}
    solution = lscov(A, b);
\end{center}
\end{enumerate}
\end{center}

\section*{Rezultat}
\begin{center}
solution = \begin{pmatrix}
     0.0952\\
    -1.3810\\
          0
\end{pmatrix}
\end{center}
\begin{center}
$x\textsubscript{1}$ = 0.0952, $x\textsubscript{2}$ = -1.3810, 
$x\textsubscript{3}$ = 0
\end{center}

\section*{Observație}
Am primit avertismentul \textcolor{red}{'A is rank deficient to within machine precision'} care indică o posibilă deficiență de rang în matricea coeficienților, ceea ce poate afecta rezultatele metodei celor mai mici pătrate utilizată în funcția lscov din MATLAB, astfel conducând la rezultate inexacte.
\end{document}
