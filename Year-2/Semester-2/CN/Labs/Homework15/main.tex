% LaTeX Template For MATH 490 @ VCU
\documentclass{article}
\usepackage{hyperref}
\usepackage{amsmath}
\usepackage{amsthm}
\usepackage{amssymb}
\usepackage{xcolor}

%%%%%%%%%% EXACT 1in MARGINS + DOUBEL SPACED %%%%%%%
% NOTE IF YOU USE 1IN MARGINS CHANGE THE FONT 	  %%
% SIZE TO 12PT IN THE FIRST LINE OF THIS DOCUMENT %%
%\linespread{2}									  %%
%\setlength{\textwidth}{6.5in}     				  %%
%\setlength{\oddsidemargin}{0in}   				  %% 
%\setlength{\evensidemargin}{0in}  				  %%
%\setlength{\textheight}{8.5in}    				  %%
%\setlength{\topmargin}{0in}      				  %%
%\setlength{\headheight}{0in}     				  %%
%\setlength{\headsep}{0in}       				  %%
%\setlength{\footskip}{.5in}    				  %%
%%%%%%%%%%%%%%%%%%%%%%%%%%%%%%%%%%%%%%%%%%%%%%%%%%%%



\begin{document}

\title{Calcul numeric - tem\u{a} de laborator}

\author{}

\date{Februarie - Mai 2024}

\maketitle              % typeset the title of the contribution


% You don't need an abstract or keywords for an article review
%\begin{abstract}
%The abstract should summarize the contents of the paper
%using at least 70 and at most 150 words. It will be set in 9-point
%font size and be inset 1.0 cm from the right and left margins.
%There will be two blank lines before and after the Abstract. \dots
%\keywords{List up to three keywords here, like this:
%computational geometry, graph theory, Hamilton cycles}
%\end{abstract}


% TO MAKE A TITLE PAGE USE THE FOLLOWING COMMAND HERE.
% \newpage




\section*{Enun\c{t}: Capitolul 12, Subcapitolul III, Problema 1}

S\u{a} se calculeze funcțiile de aproximare construite prin metoda celor mai
mici pătrate pentru metoda Levenberg-Marquardt pentru cazurile exercițiului II.
\[
\phi(t) = a \ln(bt + c)
\]

\[
y = 2 \ln(3t + 1)
\] 

\section*{Solu\c{t}ie}

\begin{center}
\begin{enumerate}
\item  Definim datele problemei. \\
\begin{center}
t = (0:9)'; \\
y = 2 * log(3 * t + 1); \\
\end{center}
\item Definim funcția model. \\
 \begin{center}
    model = @(params, t) params(1) * log(params(2) * t + params(3));
 \end{center}
\item Alegem punctele de pornire.\\
 \begin{center}
     initialParams = [2, 3, 1];
 \end{center}
 \item Alegem opțiunile pentru algoritmul Levenberg-Marquardt.\\
 \begin{center}
    options = optimoptions('lsqcurvefit', 'Algorithm', 'levenberg-marquardt', 'Display', 'off');
 \end{center}
  \item Aplicăm metoda Levenberg-Marquardt folosind lsqcurvefit\\
 \begin{center}
    [optimizedParams, resnorm, residual, exitflag, output] = lsqcurvefit(model, initialParams, t, y, [], [], options);
 \end{center}
\end{enumerate}
\end{center}

\section*{Rezultat}
\begin{center}
a = 2, b = 3, c = 1
\end{center}
\end{document}
