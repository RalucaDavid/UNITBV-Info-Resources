% LaTeX Template For MATH 490 @ VCU
\documentclass{article}
\usepackage{hyperref}
\usepackage{amsmath}
\usepackage{amsthm}
\usepackage{amssymb}
\usepackage{xcolor}

%%%%%%%%%% EXACT 1in MARGINS + DOUBEL SPACED %%%%%%%
% NOTE IF YOU USE 1IN MARGINS CHANGE THE FONT 	  %%
% SIZE TO 12PT IN THE FIRST LINE OF THIS DOCUMENT %%
%\linespread{2}									  %%
%\setlength{\textwidth}{6.5in}     				  %%
%\setlength{\oddsidemargin}{0in}   				  %% 
%\setlength{\evensidemargin}{0in}  				  %%
%\setlength{\textheight}{8.5in}    				  %%
%\setlength{\topmargin}{0in}      				  %%
%\setlength{\headheight}{0in}     				  %%
%\setlength{\headsep}{0in}       				  %%
%\setlength{\footskip}{.5in}    				  %%
%%%%%%%%%%%%%%%%%%%%%%%%%%%%%%%%%%%%%%%%%%%%%%%%%%%%



\begin{document}

\title{Calcul numeric - tem\u{a} de laborator}

\author{}

\date{Februarie - Mai 2024}

\maketitle              % typeset the title of the contribution


% You don't need an abstract or keywords for an article review
%\begin{abstract}
%The abstract should summarize the contents of the paper
%using at least 70 and at most 150 words. It will be set in 9-point
%font size and be inset 1.0 cm from the right and left margins.
%There will be two blank lines before and after the Abstract. \dots
%\keywords{List up to three keywords here, like this:
%computational geometry, graph theory, Hamilton cycles}
%\end{abstract}


% TO MAKE A TITLE PAGE USE THE FOLLOWING COMMAND HERE.
% \newpage




\section*{Enun\c{t}: Capitolul 10, Subcapitolul II, Problema 5}

S\u{a} se deducă expresia polinomului de interpolare pentru datele problemei
I:

\begin{center}
\[
f(x) = \sin(x),
\quad x_i = -\frac{\pi}{2} + i \cdot \frac{\pi}{10},
\quad i \in \{0, 1, 2, 3, 4, 5, 6, 7, 8, 9, 10\},
\quad z = \frac{\pi}{13}
\] 

\end{center}

\section*{Solu\c{t}ie}

\begin{center}
\begin{enumerate}
\item Definim funcția f.\\
 \begin{center}
    f = @(x) sin(x);
    \end{center}
\item  Definim valorile lui $x_i$. \\
\begin{center}
    xi = -pi/2 + (0:10)*pi/10;
\end{center}
\item Definim valoarea lui z. \\
 \begin{center}
    z = pi/13;
 \end{center}
\item Calculăm gradului polinomului. \\
 \begin{center}
    grad = length(xi) - 1;
 \end{center}
 \item Ajustăm polinomul de interpolare. \\
 \begin{center}
    coef = polyfit(xi, f(xi), grad);
 \end{center}
  \item Construim polinomul de interpolare. \\
 \begin{center}
    polinom = poly2sym(coef);
 \end{center}
\end{enumerate}
\end{center}

\section*{Rezultat}
$\frac{5845419439946349}{2535301200456458802993406410752} \cdot x^{10} + \frac{6192979824596231}{2361183241434822606848} \cdot x^9 - \frac{7753445309256651}{633825300114114700748351602688} \cdot x^8 - \frac{913899863490885}{4611686018427387904} \cdot x^7 + \frac{6860929148312471}{316912650057057350374175801344} \cdot x^6 + \frac{4803736910141451}{576460752303423488} \cdot x^5 - \frac{4742251131395917}{316912650057057350374175801344} \cdot x^4 - \frac{6004797759129527}{36028797018963968} \cdot x^3 + \frac{2255690891732239}{633825300114114700748351602688} \cdot x^2 + \frac{9007199225352387}{9007199254740992} \cdot x - \frac{3517712254454825}{20282409603651670423947251286016} = 0$



\section*{Observație}
A trebuit să instalez \textbf{Symbolic Math Toolbox} pentru a folosi funcția polyfit.
\end{document}
