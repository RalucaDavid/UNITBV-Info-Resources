% LaTeX Template For MATH 490 @ VCU
\documentclass{article}
\usepackage{hyperref}
\usepackage{amsmath}
\usepackage{amsthm}
\usepackage{amssymb}
\usepackage{xcolor}

%%%%%%%%%% EXACT 1in MARGINS + DOUBEL SPACED %%%%%%%
% NOTE IF YOU USE 1IN MARGINS CHANGE THE FONT 	  %%
% SIZE TO 12PT IN THE FIRST LINE OF THIS DOCUMENT %%
%\linespread{2}									  %%
%\setlength{\textwidth}{6.5in}     				  %%
%\setlength{\oddsidemargin}{0in}   				  %% 
%\setlength{\evensidemargin}{0in}  				  %%
%\setlength{\textheight}{8.5in}    				  %%
%\setlength{\topmargin}{0in}      				  %%
%\setlength{\headheight}{0in}     				  %%
%\setlength{\headsep}{0in}       				  %%
%\setlength{\footskip}{.5in}    				  %%
%%%%%%%%%%%%%%%%%%%%%%%%%%%%%%%%%%%%%%%%%%%%%%%%%%%%



\begin{document}

\title{Calcul numeric - tem\u{a} de laborator}

\author{}

\date{Februarie - Mai 2024}

\maketitle              % typeset the title of the contribution


% You don't need an abstract or keywords for an article review
%\begin{abstract}
%The abstract should summarize the contents of the paper
%using at least 70 and at most 150 words. It will be set in 9-point
%font size and be inset 1.0 cm from the right and left margins.
%There will be two blank lines before and after the Abstract. \dots
%\keywords{List up to three keywords here, like this:
%computational geometry, graph theory, Hamilton cycles}
%\end{abstract}


% TO MAKE A TITLE PAGE USE THE FOLLOWING COMMAND HERE.
% \newpage




\section*{Enun\c{t}: Capitolul 13, Subcapitolul II, Problema 5}

S\u{a} se calculeze integralele duble:
\[
\iint_{D} \frac{1}{\sqrt{x} + \sqrt{y}} \, dx \, dy \quad  \quad D : x = 1, \; y = 0, \; x - y = \frac{1}{2}
\]

\section*{Solu\c{t}ie}

\begin{center}
\begin{enumerate}
\item  Definim funcția. \\
\begin{center}
f = @(x, y) 1 ./ (sqrt(x) + sqrt(y));
\end{center}
\item Definim limitelor de integrare pentru y. \\
 \begin{center}
    yinf = @(x) x.\^{}2;\\
ysup = @(x) sqrt(x);
 \end{center}
\item Calculăm integrala duble folosind funcția integral2.\\
 \begin{center}
     I = integral2(f, 0, 1, yinf, ysup);
 \end{center}
\end{enumerate}
\end{center}

\section*{Rezultat}
\begin{center}
I = 0.2928
\end{center}
\end{document}
