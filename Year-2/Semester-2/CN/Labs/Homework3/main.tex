% LaTeX Template For MATH 490 @ VCU
\documentclass{article}
\usepackage{hyperref}
\usepackage{amsmath}
\usepackage{amsthm}
\usepackage{amssymb}

%%%%%%%%%% EXACT 1in MARGINS + DOUBEL SPACED %%%%%%%
% NOTE IF YOU USE 1IN MARGINS CHANGE THE FONT 	  %%
% SIZE TO 12PT IN THE FIRST LINE OF THIS DOCUMENT %%
%\linespread{2}									  %%
%\setlength{\textwidth}{6.5in}     				  %%
%\setlength{\oddsidemargin}{0in}   				  %% 
%\setlength{\evensidemargin}{0in}  				  %%
%\setlength{\textheight}{8.5in}    				  %%
%\setlength{\topmargin}{0in}      				  %%
%\setlength{\headheight}{0in}     				  %%
%\setlength{\headsep}{0in}       				  %%
%\setlength{\footskip}{.5in}    				  %%
%%%%%%%%%%%%%%%%%%%%%%%%%%%%%%%%%%%%%%%%%%%%%%%%%%%%



\begin{document}

\title{Calcul numeric - tem\u{a} de laborator}

\author{}

\date{Februarie - Mai 2024}

\maketitle              % typeset the title of the contribution


% You don't need an abstract or keywords for an article review
%\begin{abstract}
%The abstract should summarize the contents of the paper
%using at least 70 and at most 150 words. It will be set in 9-point
%font size and be inset 1.0 cm from the right and left margins.
%There will be two blank lines before and after the Abstract. \dots
%\keywords{List up to three keywords here, like this:
%computational geometry, graph theory, Hamilton cycles}
%\end{abstract}


% TO MAKE A TITLE PAGE USE THE FOLLOWING COMMAND HERE.
% \newpage




\section*{Enun\c{t}: Capitolul 8, Subcapitolul III, Problema 4}

S\u{a} se rezolve sistemele algebrice de ecuații liniare:
\begin{center}
\begin{align*}
\begin{cases}
x + 2y + 3z + 4t &= -4 \\
x + y + 2z + 3t &= -2 \\
x + 3y + z + 2t &= -3 \\
x + 3y + 3z + 2t &= -5 \\
 \end{cases}
\end{align*}
\end{center}

\section*{Solu\c{t}ie}

\begin{center}
\begin{enumerate}
\item Declarăm matricea coeficienților A. \\
 \begin{center}
    A=[1,2,3,4;1,1,2,3;1,3,1,2;1,3,3,2];
    \end{center}
\item Declarăm vectorul termenilor liberi b. \\
\begin{center}
    b=[-4;-2;-3;-5];
    \end{center}
\item  Concatenăm matricea coeficienților A cu vectorul termenilor liberi b, astfel rezultând matricea B.\\
\begin{center}
    B=[A,b];
    \end{center}
\item Aplicăm metoda Gauss-Jordan prin apelarea funcției rref.\\
\begin{center}
    rref\_B = rref(B);
\end{center}
\item Extragem soluția din matricea escalonată redusă.
\begin{center}
     solution = rref\_B(:,end);
\end{center}
\end{enumerate}
\end{center}

\section*{Rezultat}
\begin{center}
solution = \begin{pmatrix}
     1\\
    -1\\
    -1\\
     0
\end{pmatrix}
\end{center}
\begin{center}
x=1, y=-1, z=-1, t=0;
\end{center}
\end{document}
