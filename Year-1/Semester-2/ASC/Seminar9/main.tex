\documentclass{article}
\usepackage[utf8]{inputenc}
\usepackage{amsmath}
\usepackage{tabularx}
\usepackage{circuitikz}
\usetikzlibrary{calc}
\title{Problema 1 - seminar 9}
\author{}
\date{}

\begin{document}

\maketitle
\sloppy
\large\textbf{Enunțul: }\normalsize Să se reprezinte cu porți XOR circuitul de conversie pe 3 biți din cod Gray în cod binar.

\large\textbf{Informații necesare:}
\normalsize\\
\textbf{Poarta XOR:}
\\ Tabelul de valorile al porții XOR:
\begin{center}
\begin{tabularx}{0.8\textwidth} { 
  | >{\centering\arraybackslash}X 
  | >{\centering\arraybackslash}X 
  | >{\centering\arraybackslash}X | }
 \hline
   & 0 & 1 \\
 \hline
  0  & 0 & 1 \\
 \hline
  1 & 1 & 0 \\
\hline
\end{tabularx}
\end{center}
\text Desenul porții XOR: \\
\begin{circuitikz}
\draw (0,0) node[xor port] (xor1) {}
(xor1.in 1) node[anchor=east] {A}
(xor1.in 2) node[anchor=east] {B}
(xor1.out) node[anchor=west] {Y};
\end{circuitikz}\\
\textbf{Codul Gray pe 3 biți:}
\begin{center}
\begin{tabular}{||c c||} 
 \hline
 Cod Gray & Număr \\ [0.5ex] 
 \hline\hline
 000 & 0 \\ 
 \hline
 001 & 1 \\
 \hline
 011 & 2 \\
 \hline
 010 & 3 \\
 \hline
 110 & 4 \\ 
 \hline
 111 & 5 \\
 \hline
 101 & 6\\
 \hline
 100 & 7 \\ 
 \hline 
\end{tabular}
\end{center}
\text Sunt așezați astfel încât au un bit diferență. \\ Mai multe informații despre codul Gray: https://www.infoarena.ro/coduri-gray\\
\textbf{Pașii:}
\begin{enumerate}
  \item Identificarea funcției logice și expresiilor logice ale funcției.
  \item Completarea tabelului de valori.
  \item Reprezentarea circuitului logic.
\end{enumerate}

\large\textbf{Rezolvare:}\normalsize\\
\textbf{Funcțiile logice:}\\
\text Avem de găsit 3 funcții, fiecare pentru aflarea unei cifre a numărului în cod binar.\\ Vom nota cu:
\begin{itemize}
  \item a - prima cifră a numărului în cod Gray
  \item b - a doua cifră a numărului în cod Gray
  \item c - a treia cifră a numărului în cod Gray
\end{itemize}
Avem următorul tabel:
\begin{center}
\begin{tabular}{||c c||} 
 \hline
 Cod Gray & Cod Binar \\ [0.5ex] 
 \hline\hline
 000 & 000 \\ 
 \hline
 001 & 001 \\
 \hline
 011 & 010 \\
 \hline
 010 & 011 \\
 \hline
 110 & 100 \\ 
 \hline
 111 & 101 \\
 \hline
 101 & 110 \\
 \hline
 100 & 111 \\ 
 \hline 
\end{tabular}
\end{center}
\text Funcția 1:\\
\text Putem observa cum prima cifră este aceeași la ambele coduri.
\begin{center}
\begin{tabular}{||c c||} 
 \hline
 Cod Gray & Cod Binar \\ [0.5ex] 
 \hline\hline
 \textbf{0}00 & \textbf{0}00 \\ 
 \hline
 \textbf{0}01 & \textbf{0}01 \\
 \hline
 \textbf{0}11 & \textbf{0}10 \\
 \hline
 \textbf{0}10 & \textbf{0}11 \\
 \hline
 \textbf{1}10 & \textbf{1}00 \\ 
 \hline
 \textbf{1}11 & \textbf{1}01 \\
 \hline
 \textbf{1}01 & \textbf{1}10 \\
 \hline
 \textbf{1}00 & \textbf{1}11 \\ 
 \hline 
\end{tabular}
\end{center}
Rezultă \textbf{f1(a)=a}, unde f1 este funcția pentru determinarea primei cifre a numărului în cod binar.\\ Funcția 2:\\ Putem observa că, dacă facem XOR dintre prima cifră (a) și a doua cifră (b) a numărului în cod Gray, obținem a doua cifră a numărului în cod binar.
\begin{center}
\begin{tabular}{||c c c c||} 
\hline
 a & b & $a \oplus b $& Numărul în cod binar \\ [0.5ex] 
 \hline\hline
 0 & 0 & 0 & 0\textbf{0}0 \\ 
 \hline
 0 & 0 & 0 & 0\textbf{0}1 \\
 \hline
 0 & 1 & 1 & 0\textbf{1}0 \\
 \hline
 0 & 1 & 1 & 0\textbf{1}1 \\
 \hline
 1 & 1 & 0 & 1\textbf{0}0 \\ 
 \hline
 1 & 1 & 0 & 1\textbf{0}1 \\
 \hline
 1 & 0 & 1 & 1\textbf{1}0 \\
 \hline
 1 & 0 & 1 & 1\textbf{1}1 \\ 
 \hline 
\end{tabular}
\end{center}
\text Rezultă \textbf{f2(a,b)=a $\oplus$ b}, unde f2 este funcția pentru determinarea cifrei a doua a numărului în cod binar.\\ Funcția 3:\\Observăm că obținem a treia cifră a numărului în cod binar, dacă luăm XOR-ul făcut anterior dintre prima cifră (a) și a doua cifră (b) a numărului în cod Gray și îl folosim într-un XOR cu a treia cifră (c) a numărului în cod Gray obținem a treia cifră a numărului în cod binar.
\begin{center}
\begin{tabular}{||c c c c||} 
\hline
 $a \oplus b $ & c & $(a \oplus b) \oplus c$ & Numărul în cod binar \\ [0.5ex] 
 \hline\hline
 0 & 0 & 0 & 00\textbf{0} \\ 
 \hline
 0 & 1 & 1 & 00\textbf{1} \\
 \hline
 1 & 1 & 0 & 01\textbf{0} \\
 \hline
 1 & 0 & 1 & 01\textbf{1} \\
 \hline
 0 & 0 & 0 & 10\textbf{0} \\ 
 \hline
 0 & 1 & 1 & 10\textbf{1} \\
 \hline
 1 & 1 & 0 & 11\textbf{0} \\
 \hline
 1 & 0 & 1 & 11\textbf{1} \\ 
 \hline 
\end{tabular}
\end{center}
\text Rezultă \textbf{f3(a,b,c)=(a $\oplus$ b) $\oplus$ c}, unde f3 este funcția pentru determinarea cifrei a treia a numărului în cod binar.\\
\textbf{Tabelul de valori:}\\
\text Avem cele trei funcții obținute:
\begin{itemize}
  \item f1(a)=a
  \item f2(a,b)=a $\oplus$ b
  \item f3(a,b,c)=(a $\oplus$ b) $\oplus$ c
\end{itemize}
\begin{center}
\begin{tabular}{||c c c c c c||} 
 \hline
 a & b & c & f1 & f2 & f3 \\ [0.5ex] 
 \hline\hline
 0 & 0 & 0 & 0 & 0 & 0\\ 
 \hline
 0 & 0 & 1 & 0 & 0 & 1\\
 \hline
 0 & 1 & 1 & 0 & 1 & 0\\
 \hline
 0 & 1 & 0 & 0 & 1 & 1\\
 \hline
 1 & 1 & 0 & 1 & 0 & 0\\ 
 \hline
 1 & 1 & 1 & 1 & 0 & 1\\
 \hline
 1 & 0 & 1 & 1 & 1 & 0\\
 \hline
 1 & 0 & 0 & 1 & 1 & 1\\ 
 \hline 
\end{tabular}
\end{center}
\textbf{Circuitul:}\\
\begin{circuitikz}
    \draw (0,0) node[anchor=east] {a} to[short, o-] (2,0);
    \draw (0,-2) node[anchor=east] {b} to[short, o-] (2,-2);
    \draw (0,-4) node[anchor=east] {c}to[short, o-] (2,-4);
    \draw (2,0) -- (10,0) node[anchor=west] {f1};
    \draw (6,-1) node[xor port] (myxor) {}
        (myxor.in 1) -- ++(-1,0) |- (2,0)
        (myxor.in 2) -- ++(-1,0) |- (2,-2)
        (myxor.out) -- ++(4,0) node[anchor=west] {f2};
    \draw (9,-3) node[xor port] (myxor2) {}
        (myxor2.in 1) -- ++(-1,0) |- (myxor.out)
        (myxor2.in 2) -- ++(-1,0) |- (2,-4)
        (myxor2.out) -- ++(1,0) node[anchor=west] {f3};
\end{circuitikz}
\end{document}
